\chapter{Introduction}
\label{chap:introduction}

Conlangtionary exists to solve a very specific problem: storing a language in a web platform (as a dictionary and grammar) in such a way that you could theoretically represent any spoken language. Why does this problem matter? Many existing online dictionary services (e.g. Wiktionary \cite{Wiktionary}) make assumptions about the language that you are working on. Namely, they assume that the language already exists or that it is relevant to the entire online community. While these may seem like good assumptions, they actually hinder the platform's usefulness to two audiences: field linguists and ``conlangers".

Field linguists are simply researchers working to profile and preserve obscure spoken languages \cite{Field-linguistics}, but conlangers require more explanation. Conlanging is the hobby of inventing languages. To put it in more eloquent terms:

\begin{quote}
Conlanging is to linguistics what painting is
to art history, or hacking to computer science.
It’s a way of directly playing with language—
sometimes just for fun, and sometimes to test
out a new theory about how language works
with the mind. \cite{Conlanging-101}
\end{quote}

People conlang for many reasons. Authors often create languages for fictional worlds. TV shows and movie productions have recently started to hire conlangers to create the fictional languages for their worlds. Marc Okrand is best known for creating the Klingon language, but he has also created other languages commercially such as the Atlantean language for Disney's Atlantis. His success, and the success of others like him, has inspired many people to dabble in language creation. These cells of conlangers find one another through the internet, often via reddit's /r/conlangs or the Language Creation Society's listserv, and they build communities dedicated to sharing languages and techniques for their creation. At the time of this writing, reddit's conlanging community numbers 9,098 members. These communities share their languages online with platforms like Wikipedia that, while free and open-source, ultimately hinder their efforts by forcing users to create a page for each word and to manually enter a lot of information. Entering data in this way is tedious, and the results of such effort only serve to demonstrate how such a page-per-word format is ill-suited to the task of representing language.

To be clear, there are software tools such as PolyGlot for developing conlangs, but none of them are accessible as web applications. They force users to develop in a desktop environment, which makes it much harder for conlangers to share their languages with others or to collaborate on a common set of files. Conlangtionary is an effort to change that.

Chapter~\ref{chap:description} describes the design and intentions of the model that the project uses to store and manage user-created languages.

Chapter~\ref{chap:implementation} discusses the technologies and frameworks used to create Conglangtionary and explains why they were chosen.

Chapter~\ref{chap:usage} walks through using the Conlangtionary platform to create a language.

Chapter~\ref{chap:future} suggests improvements for future researchers to make the platform more useful.

Chapter~\ref{chap:conclusion} summarizes the project.